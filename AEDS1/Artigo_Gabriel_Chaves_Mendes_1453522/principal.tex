%%%%%%%%%%%%%%%%%%%%%%%%%%%%%%%%%%%%%%%%%%%%%%%%%%%%%%%%%%%%%%%%%%%%%%%%%%%%%%%%%%%%%%%%%%%%%%%%%%%%%%%
%%%%%%%%%%%%%% Template de Artigo Adaptado para Trabalho de Diplomação do ICEI %%%%%%%%%%%%%%%%%%%%%%%%
%% codificação UTF-8 - Abntex - Latex -  							     %%
%% Autor:    Fábio Leandro Rodrigues Cordeiro  (fabioleandro@pucminas.br)                            %% 
%% Co-autor: Prof. João Paulo Domingos Silva, Harison da Silva e Anderson Carvalho                   %%
%% Revisores normas NBR (Padrão PUC Minas): Helenice Rego Cunha e Prof. Theldo Cruz                  %%
%% Versão: 1.1     18 de dezembro 2015                     	                                     %%
%%%%%%%%%%%%%%%%%%%%%%%%%%%%%%%%%%%%%%%%%%%%%%%%%%%%%%%%%%%%%%%%%%%%%%%%%%%%%%%%%%%%%%%%%%%%%%%%%%%%%%%


\documentclass[a4paper,12pt,Times]{article}
\usepackage{abakos}  %pacote com padrão da Abakos baseado no padrão da PUC

%%%%%%%%%%%%%%%%%%%%%%%%%%%
%Capa da revista
%%%%%%%%%%%%%%%%%%%%%%%%%%

%\setcounter{page}{80} %iniciar contador de pagina de valor especificado
\newcommand{\monog}{Os três tipos lógicos de pesquisa, por Gilson Volpato}
\newcommand{\monogES}{Article template Institute of Mathematical Sciences and Informatics}
\newcommand{\tipo}{Artigo }  % Especificar a seção tipo do trabalho: Artigo, Resumo, Tese, Dociê etc
\newcommand{\origem}{Brasil }
\newcommand{\editorial}{Belo Horizonte, p. 01-11, nov. 2015}  % p. xx-xx – páginas inicial-final do artigo
\newcommand{\lcc}{\scriptsize{Licença Creative Commons Attribution-NonCommercial-NoDerivs 3.0 Unported}}

%%%%%%%%%%%%%%%%%INFORMAÇÕES SOBRE AUTOR PRINCIPAL %%%%%%%%%%%%%%%%%%%%%%%%%%%%%%%
\newcommand{\AutorA}{Gabriel Chaves Mendes}
\newcommand{\funcaoA}{}
\newcommand{\emailA}{gabriel.mendes.1453522@sga.pucminas.br}
\newcommand{\cursA}{Aluno do Programa de Graduação em Ciência da Computação}

\newcommand{\AutorB}{Theldo Cruz Franqueira}
\newcommand{\funcaoB}{}
\newcommand{\emailB}{104103@sga.pucminas.br}
\newcommand{\cursB}{Professor do Programa de Graduação em Ciência da Computação}
% 
% Definir macros para o nome da Instituição, da Faculdade, etc.
\newcommand{\univ}{Pontifícia Universidade Católica de Minas Gerais}

\newcommand{\keyword}[1]{\textsf{#1}}

\begin{document}
% %%%%%%%%%%%%%%%%%%%%%%%%%%%%%%%%%%
% %% Pagina de titulo
% %%%%%%%%%%%%%%%%%%%%%%%%%%%%%%%%%%

\begin{center}
\includegraphics[scale=0.2]{figuras/brasao.jpg} \\
PONTIFÍCIA UNIVERSIDADE CATÓLICA DE MINAS GERAIS \\
Instituto de Ciências Exatas e de Informática

% \vspace{1.0cm}

\end{center}

 \vspace{0cm} {
 \singlespacing \Large{\monog \symbolfootnote[1]{Artigo apresentado ao Instituto de Ciências Exatas e Informática da Pontifícia Universidade Católica de Minas Gerais como pré-requisito para obtenção do título de Bacharel em Ciência da Computação.} \\ }
  \normalsize{\monogES}
 }

\vspace{1.0cm}

\begin{flushright}
\singlespacing 
\normalsize{\AutorA \footnote{\funcaoA \cursA, \origem -- \emailA . }} \\
\normalsize{\AutorB \footnote{\funcaoB \cursB, \origem -- \emailB . }} \\
%\normalsize{\AutorC \footnote{\funcaoC \cursC, \origem -- \emailC . }} \\
%\normalsize{\AutorD \footnote{\funcaD \cursD, \origem -- \emailD . }} \\
%deixar com o valor `0` e usar o '*' no inicio da frase
% \symbolfootnote[0]{Artigo recebido em 10 de julho de 1983 e aprovado em 29 de maio 2012}
\end{flushright}
\thispagestyle{empty}

\vspace{1.0cm}

\begin{abstract}
\noindent
O vídeo criado por Gilson Volpato, apresenta três tipos de lógicas de pesquisa para artigos científicos, sendo eles argumentos com hispóteses e sem hipóteses, para auxiliar no desenvolver de uma pergunta ou questionamento. Este vídeo argumenta também sobre a importancia dos textos científicos como forma de expressão lógica e argumentativa, pois em resumo, ele ajuda a desempenhar um papel na construção do argumento cinetífico, por meio do uso ou não de hipóteses. Ao longo do vídeo, ele demostra que essas hipóteses são criadas durante o dia a dia, observando e analisando situações, elas podem ser associadas no desenvolver de outra hipótese ou interferir  neste processo, mas ajudando a desvendar questões que futuramente podem vir a ser argumentos científicos. Mas, em algumas situações não a há hipóteses associadas a pergunta, sendo necessário a avaliação de apenas uma variável.
\\\textbf{\keyword{Palavras-chave: }} Hipótese. Artigo. Científico
\end{abstract}

%%%%%%%%%%%%%%%%%%%%%%%%%%%%%%%%%%%%%%%%%%%%%%%%%%%%%%%%%
 \newpage    %%%% CASO QUEIRA QUE O RESUMO FIQUE EM UMA PAGINA E O ABSTRACT EM OUTRA
\selectlanguage{english}
\begin{abstract}
\noindent
The video created by Gilson Volpato presents three types of research logic for scientific articles, including arguments with hypotheses and without hypotheses, to assist in the development of a question or inquiry. This video also argues about the importance of scientific texts as a form of logical and argumentative expression, because in short, it helps to highlight a role in the construction of the kinetic argument, through the use or not of hypotheses. Throughout the video, he demonstrates that these hypotheses are created during everyday life, observing and analyzing situations, they can be associated in the development of other hypotheses or interfere in this process, but helping to uncover issues that may become scientific arguments in the future. . However, in some situations there are no hypotheses associated with the question, requiring the evaluation of only one variable.
\\\textbf{\keyword{Keywords: }} Hypothesis. Article. Scientific
\end{abstract}

\selectlanguage{brazilian}
 \onehalfspace  % espaçamento 1.5 entre linhas
 \setlength{\parindent}{1.25cm}

%%%%%%%%%%%%%%%%%%%%%%%%%%%%%%%%%%%%%%%%%%%%%%%%%
%% INICIO DO TEXTO
%%%%%%%%%%%%%%%%%%%%%%%%%%%%%%%%%%%%%%%%%%%%%%%%%

%%%%%%%%%%%%%%%%%%%%%%%%%%%%%%%%%%%%%%%%%%%%%%%%%%%%%%%%%%%%%%%%%%%%%%%%%%%%%%%%%%%%%%%%%%%%%%%%%%%%%%%
%%%%%%%%%%%%%% Template de Artigo Adaptado para Trabalho de Diplomação do ICEI %%%%%%%%%%%%%%%%%%%%%%%%
%% codificação UTF-8 - Abntex - Latex -  							     %%
%% Autor:    Fábio Leandro Rodrigues Cordeiro  (fabioleandro@pucminas.br)                            %% 
%% Co-autores: Prof. João Paulo Domingos Silva, Harison da Silva e Anderson Carvalho		     %%
%% Revisores normas NBR (Padrão PUC Minas): Helenice Rego Cunha e Prof. Theldo Cruz                  %%
%% Versão: 1.1     18 de dezembro 2015                                                               %%
%%%%%%%%%%%%%%%%%%%%%%%%%%%%%%%%%%%%%%%%%%%%%%%%%%%%%%%%%%%%%%%%%%%%%%%%%%%%%%%%%%%%%%%%%%%%%%%%%%%%%%%
\section{\esp Objetivo}

O artigo abordará sobre os três tipos de lógica de pesquisa em um artigo científico, sendo eles, argumento com hipóteses associadas, argumento com hipóteses de interferência e argumento sem hipótese, apresentados no vídeo de Gilson Volpato. Ademais, fomentar a importância que esses argumentos tem para a criação e desenvolvimento de um estudo científico e como essas hipóteses são criadas em meio ao nosso cotidiano.

\section{\esp Discussão}

O vídeo aborda como uma hipótese pode ser criada em meio ao nosso dia a dia, com uma simples pergunta pode-se argumentar de diferentes formas, usando diversas hipóteses que atuam como variáveis para tentar entender determinada situação ou questionamento. No entanto, essas hipóteses podem atuar entre elas de diferentes formas, podendo associar uma a outra e auxiliar para o entendimento do caso, ou, interferir em alguma outra hipótese, podendo desqualifica-la ou acrescentar uma informação nova a ela. Mas, também é abordado aquelas pesquisas sem a utilização de hipóteses para se fazer um estudo de casos sobre algum tema, elas atuam como hipóteses indepentes e únicas dentro de uma pesquisa, servindo para descrever estruturas e situações, ou seja, em uma pesquisa sem hipótese, é utilizado apenas uma variável para ajudar a auxiliar o estudo científico.

\section{\esp Conclusão}

A combinação de pesquisas com e sem hipóteses desempenha um papel essencial na elaboração de um artigo científico convincente. Enquanto as pesquisas com hipóteses fornecem evidências sólidas e sustentam as conclusões apresentadas pelos pesquisadores, as pesquisas sem hipóteses contextualizam o problema, exploram lacunas e destacam a relevância do tema em questão. Ao associar esses dois tipos de pesquisa, os autores podem construir um argumento abrangente que não só fundamenta suas afirmações, mas também afirmar a complexidade do tema e sugere alternativas para pesquisas futuras. Dessa forma, a inclusão tanto de pesquisas com hipótese quanto sem hipótese contribui para o avanço do conhecimento e na elaboação de uma pesquisa científica.


% \subsection{\esp Trabalhos futuros}
% 
% Sugestões de estudos posteriores são ser adicionados subseção deste capítulo de conclusão.


%%%%%%%%%%%%%%%%%%%%%%%%%%%%%%%%%%%
%% FIM DO TEXTO
%%%%%%%%%%%%%%%%%%%%%%%%%%%%%%%%%%%

% \selectlanguage{brazil}
%%%%%%%%%%%%%%%%%%%%%%%%%%%%%%%%%%%
%% Inicio bibliografia
%%%%%%%%%%%%%%%%%%%%%%%%%%%%%%%%%%%

\newpage
\singlespace{
\renewcommand\refname{REFERÊNCIAS}
\bibliographystyle{abntex2-alf}
\bibliography{bibliografia} % Nome do arquivo .bib
}

% Adicionar a referência do vídeo do YouTube
\nocite{aula-youtube}
AULA 20 de 42 - TRÊS TIPOS LÓGICOS DE PESQUISA

\end{document}


