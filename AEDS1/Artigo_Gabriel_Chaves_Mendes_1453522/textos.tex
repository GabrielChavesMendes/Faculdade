%%%%%%%%%%%%%%%%%%%%%%%%%%%%%%%%%%%%%%%%%%%%%%%%%%%%%%%%%%%%%%%%%%%%%%%%%%%%%%%%%%%%%%%%%%%%%%%%%%%%%%%
%%%%%%%%%%%%%% Template de Artigo Adaptado para Trabalho de Diplomação do ICEI %%%%%%%%%%%%%%%%%%%%%%%%
%% codificação UTF-8 - Abntex - Latex -  							     %%
%% Autor:    Fábio Leandro Rodrigues Cordeiro  (fabioleandro@pucminas.br)                            %% 
%% Co-autores: Prof. João Paulo Domingos Silva, Harison da Silva e Anderson Carvalho		     %%
%% Revisores normas NBR (Padrão PUC Minas): Helenice Rego Cunha e Prof. Theldo Cruz                  %%
%% Versão: 1.1     18 de dezembro 2015                                                               %%
%%%%%%%%%%%%%%%%%%%%%%%%%%%%%%%%%%%%%%%%%%%%%%%%%%%%%%%%%%%%%%%%%%%%%%%%%%%%%%%%%%%%%%%%%%%%%%%%%%%%%%%
\section{\esp Objetivo}

O artigo abordará sobre os três tipos de lógica de pesquisa em um artigo científico, sendo eles, argumento com hipóteses associadas, argumento com hipóteses de interferência e argumento sem hipótese, apresentados no vídeo de Gilson Volpato. Ademais, fomentar a importância que esses argumentos tem para a criação e desenvolvimento de um estudo científico e como essas hipóteses são criadas em meio ao nosso cotidiano.

\section{\esp Discussão}

O vídeo aborda como uma hipótese pode ser criada em meio ao nosso dia a dia, com uma simples pergunta pode-se argumentar de diferentes formas, usando diversas hipóteses que atuam como variáveis para tentar entender determinada situação ou questionamento. No entanto, essas hipóteses podem atuar entre elas de diferentes formas, podendo associar uma a outra e auxiliar para o entendimento do caso, ou, interferir em alguma outra hipótese, podendo desqualifica-la ou acrescentar uma informação nova a ela. Mas, também é abordado aquelas pesquisas sem a utilização de hipóteses para se fazer um estudo de casos sobre algum tema, elas atuam como hipóteses indepentes e únicas dentro de uma pesquisa, servindo para descrever estruturas e situações, ou seja, em uma pesquisa sem hipótese, é utilizado apenas uma variável para ajudar a auxiliar o estudo científico.

\section{\esp Conclusão}

A combinação de pesquisas com e sem hipóteses desempenha um papel essencial na elaboração de um artigo científico convincente. Enquanto as pesquisas com hipóteses fornecem evidências sólidas e sustentam as conclusões apresentadas pelos pesquisadores, as pesquisas sem hipóteses contextualizam o problema, exploram lacunas e destacam a relevância do tema em questão. Ao associar esses dois tipos de pesquisa, os autores podem construir um argumento abrangente que não só fundamenta suas afirmações, mas também afirmar a complexidade do tema e sugere alternativas para pesquisas futuras. Dessa forma, a inclusão tanto de pesquisas com hipótese quanto sem hipótese contribui para o avanço do conhecimento e na elaboação de uma pesquisa científica.


% \subsection{\esp Trabalhos futuros}
% 
% Sugestões de estudos posteriores são ser adicionados subseção deste capítulo de conclusão.
