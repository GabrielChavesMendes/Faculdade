%%%%%%%%%%%%%%%%%%%%%%%%%%%%%%%%%%%%%%%%%%%%%%%%%%%%%%%%%%%%%%%%%%%%%%%%%%%%%%%%%%%%%%%%%%%%%%%%%%%%%%%
%%%%%%%%%%%%%% Template de Artigo Adaptado para Trabalho de Diplomação do ICEI %%%%%%%%%%%%%%%%%%%%%%%%
%% codificação UTF-8 - Abntex - Latex -  							     %%
%% Autor:    Fábio Leandro Rodrigues Cordeiro  (fabioleandro@pucminas.br)                            %% 
%% Co-autor: Prof. João Paulo Domingos Silva, Harison da Silva e Anderson Carvalho                   %%
%% Revisores normas NBR (Padrão PUC Minas): Helenice Rego Cunha e Prof. Theldo Cruz                  %%
%% Versão: 1.1     18 de dezembro 2015                     	                                     %%
%%%%%%%%%%%%%%%%%%%%%%%%%%%%%%%%%%%%%%%%%%%%%%%%%%%%%%%%%%%%%%%%%%%%%%%%%%%%%%%%%%%%%%%%%%%%%%%%%%%%%%%


\documentclass[a4paper,12pt,Times]{article}
\usepackage{abakos}  %pacote com padrão da Abakos baseado no padrão da PUC

%%%%%%%%%%%%%%%%%%%%%%%%%%%
%Capa da revista
%%%%%%%%%%%%%%%%%%%%%%%%%%

%\setcounter{page}{80} %iniciar contador de pagina de valor especificado
\newcommand{\monog}{Modelo de artigo do Instituto de Ciências Exatas e de Informática}
\newcommand{\monogES}{Article template Institute of Mathematical Sciences and Informatics}
\newcommand{\tipo}{Artigo }  % Especificar a seção tipo do trabalho: Artigo, Resumo, Tese, Dociê etc
\newcommand{\origem}{Brasil }
\newcommand{\editorial}{Belo Horizonte, p. 16-10, out. 2024}  % p. xx-xx – páginas inicial-final do artigo
\newcommand{\lcc}{\scriptsize{Licença Creative Commons Attribution-NonCommercial-NoDerivs 3.0 Unported}}

%%%%%%%%%%%%%%%%%INFORMAÇÕES SOBRE AUTOR PRINCIPAL %%%%%%%%%%%%%%%%%%%%%%%%%%%%%%%
\newcommand{\AutorA}{Gabriel Chaves Mendes}
\newcommand{\funcaoA}{}
\newcommand{\emailA}{gabrielchavesmendes@pucminas.br}
\newcommand{\cursA}{Aluno do Programa de Graduação em Ciência da Computação}

\newcommand{\AutorB}{Theldo Cruz}
\newcommand{\funcaoB}{}
\newcommand{\emailB}{@pucminas.br}
\newcommand{\cursB}{Professor do Programa de Graduação em Ciência da Computação}
% 
% Definir macros para o nome da Instituição, da Faculdade, etc.
\newcommand{\univ}{Pontifícia Universidade Católica de Minas Gerais}

\newcommand{\keyword}[1]{\textsf{#1}}

\begin{document}
% %%%%%%%%%%%%%%%%%%%%%%%%%%%%%%%%%%
% %% Pagina de titulo
% %%%%%%%%%%%%%%%%%%%%%%%%%%%%%%%%%%

\begin{center}
\includegraphics[scale=0.2]{figuras/brasao.jpg} \\
PONTIFÍCIA UNIVERSIDADE CATÓLICA DE MINAS GERAIS \\
Instituto de Ciências Exatas e de Informática

% \vspace{1.0cm}

\end{center}

 \vspace{0cm} {
 \singlespacing \Large{\monog \symbolfootnote[1]{Artigo apresentado ao Instituto de Ciências Exatas e Informática da Pontifícia Universidade Católica de Minas Gerais como pré-requisito para obtenção do título de Bacharel em Ciência da Computação.} \\ }
  \normalsize{\monogES}
 }

\vspace{1.0cm}

\begin{flushright}
\singlespacing 
\normalsize{\AutorA \footnote{\funcaoA \cursA, \origem -- \emailA . }} \\
\normalsize{\AutorB \footnote{\funcaoB \cursB, \origem -- \emailB . }} \\
%\normalsize{\AutorC \footnote{\funcaoC \cursC, \origem -- \emailC . }} \\
%\normalsize{\AutorD \footnote{\funcaD \cursD, \origem -- \emailD . }} \\
%deixar com o valor `0` e usar o '*' no inicio da frase
% \symbolfootnote[0]{Artigo recebido em 10 de julho de 1983 e aprovado em 29 de maio 2012}
\end{flushright}
\thispagestyle{empty}

\vspace{1.0cm}

\begin{abstract}
\noindent
Este artigo apresenta uma visão geral sobre diferentes tipos de circuitos integrados, incluindo ASICs, ASSPs, SPLDs, CPLDs, SoCs e FPGAs. A comparação entre essas tecnologias destaca suas características, aplicações e limitações. Os ASICs são circuitos projetados para uma função específica, enquanto os ASSPs oferecem maior versatilidade em múltiplas aplicações. Os SPLDs são dispositivos simples, adequados para lógicas básicas, enquanto os CPLDs oferecem maior complexidade e flexibilidade. O conceito de SoC é explorado como uma integração de múltiplos componentes em um único chip, enquanto os FPGAs se destacam pela sua programabilidade pós-fabricação, permitindo adaptações a diferentes necessidades. Através da análise dessas tecnologias, discutimos as tendências atuais e futuras no design de circuitos integrados, suas aplicações em diversas indústrias e suas implicações para o desenvolvimento tecnológico.
\\\textbf{\keyword{Palavras-chave: }} Circuitos Integrados. ASIC. FPGA. Programabilidade. Desempenho Energético.
\end{abstract}

%%%%%%%%%%%%%%%%%%%%%%%%%%%%%%%%%%%%%%%%%%%%%%%%%%%%%%%%%
 \newpage    %%%% CASO QUEIRA QUE O RESUMO FIQUE EM UMA PAGINA E O ABSTRACT EM OUTRA
\selectlanguage{english}
\begin{abstract}
\noindent
This article presents an overview of different types of integrated circuits, including ASICs, ASSPs, SPLDs, CPLDs, SoCs, and FPGAs. The comparison between these technologies highlights their characteristics, applications, and limitations. ASICs are circuits designed for a specific function, while ASSPs offer greater versatility in multiple applications. SPLDs are simple devices, suitable for basic logic, while CPLDs offer greater complexity and flexibility. The SoC concept is explored as an integration of multiple components on a single chip, while FPGAs stand out for their post-fabrication programmability, allowing adaptations to different needs. Through the analysis of these technologies, we discuss current and future trends in integrated circuit design, their applications in various industries, and their implications for technological development.
\\\textbf{\keyword{Keywords: }} Integrated Circuits. ASIC. FPGA. Programmability. Energy Performance.
\end{abstract}

\selectlanguage{brazilian}
 \onehalfspace  % espaçamento 1.5 entre linhas
 \setlength{\parindent}{1.25cm}

%%%%%%%%%%%%%%%%%%%%%%%%%%%%%%%%%%%%%%%%%%%%%%%%%
%% INICIO DO TEXTO
%%%%%%%%%%%%%%%%%%%%%%%%%%%%%%%%%%%%%%%%%%%%%%%%%

%%%%%%%%%%%%%%%%%%%%%%%%%%%%%%%%%%%%%%%%%%%%%%%%%%%%%%%%%%%%%%%%%%%%%%%%%%%%%%%%%%%%%%%%%%%%%%%%%%%%%%%
%%%%%%%%%%%%%% Template de Artigo Adaptado para Trabalho de Diplomação do ICEI %%%%%%%%%%%%%%%%%%%%%%%%
%% codificação UTF-8 - Abntex - Latex -  							     %%
%% Autor:    Fábio Leandro Rodrigues Cordeiro  (fabioleandro@pucminas.br)                            %% 
%% Co-autores: Prof. João Paulo Domingos Silva, Harison da Silva e Anderson Carvalho		     %%
%% Revisores normas NBR (Padrão PUC Minas): Helenice Rego Cunha e Prof. Theldo Cruz                  %%
%% Versão: 1.1     18 de dezembro 2015                                                               %%
%%%%%%%%%%%%%%%%%%%%%%%%%%%%%%%%%%%%%%%%%%%%%%%%%%%%%%%%%%%%%%%%%%%%%%%%%%%%%%%%%%%%%%%%%%%%%%%%%%%%%%%
\section{\esp Objetivo}

O artigo abordará sobre os três tipos de lógica de pesquisa em um artigo científico, sendo eles, argumento com hipóteses associadas, argumento com hipóteses de interferência e argumento sem hipótese, apresentados no vídeo de Gilson Volpato. Ademais, fomentar a importância que esses argumentos tem para a criação e desenvolvimento de um estudo científico e como essas hipóteses são criadas em meio ao nosso cotidiano.

\section{\esp Discussão}

O vídeo aborda como uma hipótese pode ser criada em meio ao nosso dia a dia, com uma simples pergunta pode-se argumentar de diferentes formas, usando diversas hipóteses que atuam como variáveis para tentar entender determinada situação ou questionamento. No entanto, essas hipóteses podem atuar entre elas de diferentes formas, podendo associar uma a outra e auxiliar para o entendimento do caso, ou, interferir em alguma outra hipótese, podendo desqualifica-la ou acrescentar uma informação nova a ela. Mas, também é abordado aquelas pesquisas sem a utilização de hipóteses para se fazer um estudo de casos sobre algum tema, elas atuam como hipóteses indepentes e únicas dentro de uma pesquisa, servindo para descrever estruturas e situações, ou seja, em uma pesquisa sem hipótese, é utilizado apenas uma variável para ajudar a auxiliar o estudo científico.

\section{\esp Conclusão}

A combinação de pesquisas com e sem hipóteses desempenha um papel essencial na elaboração de um artigo científico convincente. Enquanto as pesquisas com hipóteses fornecem evidências sólidas e sustentam as conclusões apresentadas pelos pesquisadores, as pesquisas sem hipóteses contextualizam o problema, exploram lacunas e destacam a relevância do tema em questão. Ao associar esses dois tipos de pesquisa, os autores podem construir um argumento abrangente que não só fundamenta suas afirmações, mas também afirmar a complexidade do tema e sugere alternativas para pesquisas futuras. Dessa forma, a inclusão tanto de pesquisas com hipótese quanto sem hipótese contribui para o avanço do conhecimento e na elaboação de uma pesquisa científica.


% \subsection{\esp Trabalhos futuros}
% 
% Sugestões de estudos posteriores são ser adicionados subseção deste capítulo de conclusão.


%%%%%%%%%%%%%%%%%%%%%%%%%%%%%%%%%%%
%% FIM DO TEXTO
%%%%%%%%%%%%%%%%%%%%%%%%%%%%%%%%%%%

% \selectlanguage{brazil}
%%%%%%%%%%%%%%%%%%%%%%%%%%%%%%%%%%%
%% Inicio bibliografia
%%%%%%%%%%%%%%%%%%%%%%%%%%%%%%%%%%%

 \newpage
\singlespace{
\renewcommand\refname{REFERÊNCIAS}
\bibliographystyle{abntex2-alf}
\cite{asic}
\cite{cpld}
\cite{logica}
\bibliography{bibliografia}


}

\end{document}


